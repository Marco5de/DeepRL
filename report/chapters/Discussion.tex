\chapter{Discussion}\label{ch:discussion}
The primary goal of this work was to replicate the results achieved by the original PPO authors.
Thus, in the following I firstly discuss to what extent I was successful in reproducing the results.
Afterwards, I discuss the topic of reproducibility in science in general, and my experience in replicating the results

\section{Results}\label{sec:disc_results} %todo: besseren titel


\section{Reproducibility in Science}\label{sec:disc_repro}
In science almost all work is related to previous publications either by improving established work
or comparing novel methods to the current state of the art.
Thus, in order to compare your own results to the existing publications' reproducibility is extremely important.
In the context of machine learning and reinforcement learning this means that an author can validate the results of
others by running the experiments on their own machine.
However, not all authors publish their implementation which requires re-implementing their work based on the provided paper.
Many papers are relatively vague regarding implementation details and are missing vital information.
Thus, replicating results solely based on the paper is often difficult.
There are platforms helping with this problem.
One such platform is \textit{paperswithcode}\footnote{\url{https://paperswithcode.com/}}
grouping publications based on the task or dataset and offering links to available implementations.
This demonstrates some difficulties in regard to reproducibility in the field of machine learning.
Reinforcement learning introduces additional challenges as the entire framework presented in Section~\ref{sec:reinforcement-learning-basics}
is intrinsically based on statistics which further makes replicating exact results difficult.\\ % todo: hier vllt ausführlicher und besserausformulieren

In the following I'd like to discuss my personal experience reproducing the results of the PPO algorithm.
Previous to this project I had no practical experience in the field of reinforcement learning except a brief introduction in a single lecture.
So the beginning included a lot of learning about the basics of reinforcement learning, common conventions and an overview of the field.
Starting with the environments I found it relatively easy to setup (..) compatible with the existing work . %todo formulierung grausig
Something I struggled with especially at the beginning of the implementation was the transfer from the formulas given in
the paper to an actual implementation, e.g.\ replacing expectations by their empirical, sampled-based estimator.
The PPO paper is vague regarding the modeling of the policy and value function.
The original paper suggests the actor-critic method, it is however unclear if it is implemented as separated networks
or with parameter sharing.
Available implementations also vary widely in this point.
Once I had a working implementation, I had some problems with hyperparameters.
To achieve optimal performance I tried to use the same hyperparameters as available implementations but depending on the
implementation they change their name and also their definition which makes it difficult to copy the hyperparameters
when one is not intimately familiar with the common conventions.\\
Even though I had no previous experience with reinforcement learning, I have some experience in Deep Learning
regarding reproducibility from my bachelor thesis and various projects at Team Spatzenhirn.
For me, it was significantly more difficult reproducing the results of the PPO algorithm compared to previous projects
I worked on such as monocular depth estimation or depth estimation.
Something I noticed is that especially when transferring to a new dataset / environment the reinforcement learning
models are incredibly sensitive w.r.t.\ their hyperparameters.
I imagine that this makes it difficult to transfer a model from one of the popular dataset to an actual application. % todo dieses Ende mit bezug auf meine bisherige Erfahrung besser ausformulieren!
